\documentclass[a4paper, 12pt]{article}
\usepackage[T1]{fontenc}
\usepackage[utf8]{inputenc}
\usepackage[francais]{babel}
\pagestyle{headings}

\title{Rapport du projet C Learning Environment}
\author{Jérémy Domarin \and Sanaa El Haouzi \and Houssem Aloui}
\begin{document}
\maketitle
\tableofcontents

\newpage

\section{Logistique}
\subsection{Outils utilisés}
Nous avons utilisé l'IDE Geany pour le codage et pour le fichier source qui a généré ce rapport. Le projet a été entièrement codé et fonctionne sous Linux. Nous n'avons pas modifié l'interface graphique, considérant qu'il s'agit avant tout d'un projet de système. 
Nous avons aussi utilisé l'outil GDB pour le débogage et notre projet était versionné : nous avons crée un dépôt SVN sur la forge de l'ESIAL.
\subsection{Temps passé}
Jérémy Domarin : \newline
\tabularnewline codage : \newline
\tabularnewline tests : \newline
\tabularnewline rapport : \newline
 
Sanaa El Haouzi : \newline
\tabularnewline codage : \newline
\tabularnewline tests : \newline
\tabularnewline rapport : \newline

Houssem Aloui : \newline
\tabularnewline codage : \newline 
\tabularnewline tests : \newline
\tabularnewline rapport : \newline
\section{Stratégie de travail}
Comme nous n'avons pas assez de recul en programmation système, et que le sujet du projet de cette année est assez ambitieux, nous avons pris le parti de suivre à la lettre la stratégie décrite dans le sujet.
\subsection{Compilation}
Dans cette partie, il faut écrire le code des élèves dans un fichier temporaire, dans lequel il doit y avoir les prototypes des fonctions que l'élève a le droit d'écrire. Après on écrit le fichier (dans /tmp) et ensuite on crée un processus qui va compiler le fichier temporaire. On fait exécuter gcc par le nouveau processus.

Appels système utilisés: \newline
	* \texttt{mkstemp}: créer un fichier temporaire; \newline
	* \texttt{fdopen} (renvoi un pointeur vers un fichier): ouvrir le fichier temporaire à partir d'un descripteur de fichier; \newline
	* \texttt{pipe}: créer un tube pour faire communiquer deux processus (père et fils); \newline
	* \texttt{fork}: dupliquer les processus; \newline
	* \texttt{close}: fermer le descipteur de fichier; \newline
	* \texttt{dup2}: rediriger la sortie du fils vers le père; \newline
	* \texttt{execlp}: faire exécuter gcc par le processus fils; \newline
	* \texttt{read}: pour que le père lise la sortie d'erreur du fils; \newline
	
\emph{Remarque}: Dans le processus fils on exécute gcc (qui compile le fichier temporaire) et dans le père on lance un nouveau thread pour faire bouger la tortue.
\subsection{Faire un binaire du code des élèves}
Dans le fichier turtle\_userside.c on définit les fonctions que les élèves ont le droit d'écrire et on fait en sorte que ce code soit compilé avec le code des étudiants.

Appels système utilisés: les mêmes que dans la première phase.
\subsection{Exécution}
Pour ce faire, nous nous sommes inspirés du code qui permet de lancer la démo, nous l'avons copié dans la méthode exercise\_run comme décrit dans le sujet.
Dans ce code nous invoquons la fonction turtle\_set\_code qui injecte le code dans la tortue.
Nous passons en paramètre de cette fonction, la fonction exercise\_run\_one que nous avons dû créer.
La méthode exercise\_run\_one permet d'exécuter le code compilé à l'étape précédente (compilation).
Nous avons utilisé les appels systèmes suivants : \texttt{fork}, \texttt{execlp}, \texttt{fdopen} plus : \newline
* \texttt{getline} qui permet de lire un flux de commande ligne par ligne ;\newline
* \texttt{kill} : nous en avons besoin pour implémentation du bouton stop.\newline
\emph{Remarque}: pour le bouton stop nous avons utilisé un mutex qui assure le verrouillage de la structure de données qui stocke les pid des processus tués.
\subsection{Interaction avec le monde}
Nous avons effectué quelques modifications sur le fichier turtle\_userside.c et avons lancé une boucle pour lire l'affichage du processus fils. Ensuite nous appelons les fonctions simulant le comportement de la tortue dans un switch pour exécuter le bon code.\newline
Nous avons utilisé les appels système suivants pour cette partie : \texttt{fdopen}, \texttt{getline}, \texttt{pipe}, \texttt{dup2}, \texttt{close}.
\subsection{Finition des bases}
Le but de cette partie est de vérifier que le monde objectif et le monde des élèves est le même en utilisant la méthode world\_eq() et afficher une fenêtre de dialogue notifiant si l'élève a réussi l'exercice ou pas.
\subsection{Chargement des leçons}
Dans cette partie il nous fallait compléter la fonction lesson\_from\_file du fichier lesson.c afin de charger dynamiquement les plugins qui permettent d'accéder à d'autres exercices et d'autres leçons.
\section{Extensions}
Nous n'avons pu implémenter les extensions faute de temps et à cause des grandes difficultés qu'ont causé la programmation du c\oe ur de l'application.

\section{Difficultés rencontrées}
Nous avons eu des difficultés à bien comprendre le sujet, qui, malgré les détails qui sont mentionnés, pouvait à certains endroits prêter à confusion. 
Nous savions dans les grandes lignes ce qu'il fallait faire, mais ne trouvions pas les moyens techniques d'y parvenir. C'est pourquoi nous nous sommes fait aider (voir la section Remerciements).
Durant le codage, nous avons fait face à de nombreux obstacles : erreurs de compilation, erreurs de segmentation, interface graphique ne répondant plus lorsqu'on appuie sur le bouton Run, etc.
Nous nous sommes rendus compte en programmant que l'organisation des fichiers d'include (ceux dont les noms sont précédés de \#include) pouvait différer suivant la distribution du système d'exploitation.
\section{Conclusion}

\appendix
\section{Remerciements}
Nous avons utilisé les ressources des sites web suivants pour réaliser notre projet :\newline
http://rangiroa.essi.fr/cours/systeme1/02-posix\_threads.pdf : Cours sur les threads ; \newline
http://www.clubic.com/ : Autre site d'aide à la programmation \newline
http://www.generation-nt.com/ : idem \newline
http://www.cplusplus.com/reference : le site de référence du langage C et C++ ;\newline
Le site du Zéro. \newline
Forums d'Ubuntu, de Linux. \newline
Le site de Comment ça marche. \newline
La documentation en ligne de GTK. \newline

Nous nous sommes fait aider par les personnes suivantes : \newline
* Damien Schirm (ESIAL 2\ieme\ année) pour le paragraphe 3.2.4. \newline
 
Nous avons réalisé ce rapport en utilisant \LaTeX.

\end{document}
